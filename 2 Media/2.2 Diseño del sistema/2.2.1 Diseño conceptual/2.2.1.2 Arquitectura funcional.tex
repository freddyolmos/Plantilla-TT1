\subsection{Arquitectura funcional}
La arquitectura funcional del sistema se ilustra mediante una estructura FBS

\renewcommand{\arraystretch}{0.7}
\begin{table}[H]
	\centering
	\caption{Funciones y subfunciones del Sistema Mecatrónico de Entrenamiento}
	\begin{tabular}{l}
		\toprule
\textbf{Función principal (A-0): Configurar Cubo para Entrenamiento F2L} \\ \midrule

1.0 \; Gestionar energía \\ 
\hspace{0.5cm} 1.1 \; Obtener energía \\
\hspace{0.5cm} 1.2 \; Acondicionar energía \\
\hspace{0.5cm} 1.3 \; Distribuir energía \\ \midrule

2.0 \; Gestionar información y control \\ 
\hspace{0.5cm} 2.1 \; Interactuar con el usuario \\
\hspace{1cm} 2.1.1 \; Recibir selección de caso F2L \\
\hspace{1cm} 2.1.2 \; Mostrar estado del sistema \\
\hspace{0.5cm} 2.2 \; Gestionar modelo lógico del cubo \\
\hspace{1cm} 2.2.1 \; Inicializar estado resuelto \\
\hspace{1cm} 2.2.2 \; Actualizar estado del cubo \\
\hspace{0.5cm} 2.3 \; Planificar secuencia de movimientos \\
\hspace{1cm} 2.3.1 \; Obtener algoritmo objetivo \\
\hspace{1cm} 2.3.2 \; Calcular algoritmo inverso \\
\hspace{1cm} 2.3.3 \; Traducir algoritmo a comandos \\ \midrule

3.0 \; Manipular físicamente el cubo \\ 
\hspace{0.5cm} 3.1 \; Sujetar el cubo \\
\hspace{0.5cm} 3.2 \; Rotar caras del cubo \\
\hspace{0.5cm} 3.3 \; Liberar el cubo \\ \midrule

4.0 \; Proteger y soportar el sistema \\ 
\hspace{0.5cm} 4.1 \; Soportar actuadores \\
\hspace{0.5cm} 4.2 \; Proteger componentes electrónicos \\
\hspace{0.5cm} 4.3 \; Garantizar seguridad del usuario \\ \midrule

5.0 \; Supervisar operación y seguridad \\ 
\hspace{0.5cm} 5.1 \; Detectar atascos mecánicos \\
\hspace{0.5cm} 5.2 \; Ejecutar paro de emergencia \\


\bottomrule
\end{tabular}
\end{table}
\renewcommand{\arraystretch}{1.0}

\begin{figure}[H]
    \centering
    \includegraphics[height=\textwidth,angle=90]{img/idef-0-cubo.pdf} % o .png
    \caption{Diagrama IDEF-0 del cubo}
    \label{fig:idef0cubo}
\end{figure}
