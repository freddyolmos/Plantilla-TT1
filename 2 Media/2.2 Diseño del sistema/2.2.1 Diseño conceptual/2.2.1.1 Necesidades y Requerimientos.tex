\subsection{Necesidades y requerimientos}

Las necesidades son especificaciones que da el usuario y se deben de cubrir de forma satisfactoria, mientras
que los requerimientos se definen como especificaciones puntuales de un sistema, que incluyen una
declaración de valor y unidades.\\

\begin{table}[h]
	\centering
	\caption{Tabla de necesidades}
	\begin{tabular}{clc}
		\toprule
		\textbf{No.} & \textbf{Necesidad} & \textbf{Clasificación} \\ \midrule
		N\textsubscript{1} &   Ser eficiente    &       Funcional        \\
        N\textsubscript{2} &   Modular          &       Funcional        \\
        N\textsubscript{2} &   Que las partes moviles no se traben          &       Funcional        \\
		N\textsubscript{3} &   Preparación automática de casos CFOP    &       Funcional        \\
        N\textsubscript{4} &   Verificación por visión artificial (antes/después)    &       Funcional        \\
        N\textsubscript{5} &   Interfaz Hombre-Máquina (IHM)    &       Funcional        \\
		N\textsubscript{6} &   Gestión de base de datos de algoritmos       &      Funcional      \\
        N\textsubscript{7} &   Repetibilidad y precisión    &      No funcional      \\
        N\textsubscript{7} &   Seguridad eléctrica y mecánica    &      No funcional      \\
        N\textsubscript{7} &   elaborado con componentes comerciales    &      No funcional      \\
        N\textsubscript{7} &   Fácil operación    &      No funcional      \\
        N\textsubscript{7} &   Bajo costo    &      No funcional      \\ \bottomrule
	\end{tabular}
\end{table}

\FloatBarrier
%\subsection{Requerimientos}
Se convierten las necesidades en requerimientos medibles

\begin{table}[h]
    \scriptsize
    \centering
    \caption{Tabla de Requisitos del Sistema Propuesto}
    \label{tab:requisitos_sistema}
    \begin{tabular}{p{5mm} p{95mm} p{30mm}}
        \toprule
        \textbf{No.} & \textbf{Requerimiento} & \textbf{Rango} \\ \midrule

        R\textsubscript{01} & El sistema debe configurar automáticamente casos de entrenamiento F2L a partir de un cubo resuelto. & $\geq$ 10 casos \\

        R\textsubscript{02} & El robot debe reproducir un mismo caso sin variaciones que afecten la práctica del usuario. & Error angular $\leq \pm 2^\circ$ \\

        R\textsubscript{03} & El tiempo máximo para configurar un caso no debe exceder el límite establecido. & $\leq$ 3 min \\

        R\textsubscript{04} & El sistema debe operar sin provocar daños físicos al cubo. & $\geq$ 50 ciclos \\

        R\textsubscript{05} & El sistema debe contar con un mecanismo de referencia para cada eje. & Error $\leq 1^\circ$ \\

        R\textsubscript{06} & El robot debe detectar atascos mecánicos y detenerse automáticamente. & $\leq$ 500 ms \\

        R\textsubscript{07} & El sistema debe soportar operación continua sin sobrecalentamiento. & $\geq$ 60 min \\

        R\textsubscript{08} & La estructura mecánica debe minimizar vibraciones que afecten la precisión del sistema. & Valor a definir \\

        R\textsubscript{09} & El sistema debe incorporar un mecanismo de paro de emergencia. & Activación inmediata \\

        R\textsubscript{10} & Las partes móviles no deben representar riesgo para el usuario. & Sin atrapamientos \\

        R\textsubscript{11} & El robot debe permitir control mediante una aplicación móvil. & BLE o WiFi \\

        R\textsubscript{12} & La aplicación debe registrar métricas de entrenamiento. & $\geq$ 4 variables \\

        R\textsubscript{13} & El usuario debe poder seleccionar manualmente el caso a practicar. & 100\% funcional \\

        R\textsubscript{14} & El sistema debe almacenar el historial de sesiones. & $\geq$ 50 sesiones \\

        R\textsubscript{15} & El costo total del prototipo no debe exceder el presupuesto definido. & $\leq$ \$7,000 MXN \\

        \bottomrule
    \end{tabular}
\end{table}
\FloatBarrier



\begin{table}[h]
	\centering
	\caption{Tabla de restricciones}
	\begin{tabular}{p{10mm} p{120mm}}
		\toprule
		\textbf{No.} & \textbf{Necesidad}  \\ \midrule
		Re\textsubscript{1} &   Compatibilidad del Cubo: El diseño mecánico debe ser compatible con cubos de velocidad estándar de 3x3x3, con dimensiones de 56mm ± 5mm.            \\
         \bottomrule
	\end{tabular}
\end{table}


