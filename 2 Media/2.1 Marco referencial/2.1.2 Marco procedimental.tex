%% Metodologia mecatronica

\section{Marco procedimental}

\subsection{Definición de la metodología}

El presente trabajo toma como referencia la directriz VDI-2206 para el diseño de
sistemas mecatrónicos \cite{gausemeier2003new}, debido a que permite integrar de
manera iterativa y multidisciplinaria los dominios mecánico, electrónico, de control e
informático. Este enfoque organiza el desarrollo en forma de ``V'': se parte de las
necesidades y requerimientos del sistema, se desciende al diseño por dominios y se
asciende a la integración, verificación y validación del sistema completo.

En el contexto del robot entrenador CFOP, la VDI-2206 permite mantener trazabilidad
entre necesidad, requerimiento, arquitectura, implementación y evidencia de cumplimiento.
La Figura~\ref{fig:vdi2206} muestra el modelo general que sirve como guía metodológica
del proyecto.

\begin{figure}[h]
\centering
%\includegraphics[width=0.8\textwidth]{img/vdi-2206.png}
\caption{VDI-2206: modelo general para el diseño de sistemas mecatrónicos.}
\label{fig:vdi2206}
\end{figure}

\begin{itemize}
    \item \textbf{Diseño del sistema}: establece el concepto solución y define la
    arquitectura funcional y física del producto.
    \item \textbf{Diseño del dominio específico}: detalla los subsistemas por disciplina
    (mecánica, electrónica, control e informática) con análisis y cálculos de soporte.
    \item \textbf{Integración del sistema}: acopla los resultados de cada dominio para
    obtener un sistema global coherente y operable.
    \item \textbf{Validación y verificación}: compara el desempeño del prototipo contra
    los requerimientos definidos.
    \item \textbf{Modelado y análisis}: usa herramientas asistidas por computadora para
    simular, ajustar y justificar decisiones de diseño.
\end{itemize}

\subsection{Diagramas FBS}

\subsection{Metodología IDEF-0}

\subsection{Lenguaje Unificado de Modelado (UML)}
