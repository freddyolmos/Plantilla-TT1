\section{Marco teórico}


\subsection{Definición de la metodología}

El presente trabajo utilizara como referencia la metodología del modelo sistemático para sistemas mecatrónicos ``V'' o  VDI-2206. [\cite{gausemeier2003new}] dado que permite retroalimentación continua e integra de manera multidisciplinaria. Este modelo que sigue una secuencia en V (de ahí su nombre) que inicia con las necesidades del cliente o del proyecto y finaliza con la producción para la manufactura. El trayecto de este proceso está dividido en etapas o módulos que al final serán integrados en uno solo, trabajando sinérgicamente para cumplir las necesidades y requerimientos, el cual consta d la siguientes etapas Figura ?  \\


\begin{figure}[h]
%\includegraphics[width=8cm]{VDI.PNG}
\centering
\caption{VDI-2206 Modelo general para diseño de sistemas mecatrónicos}
\end{figure}\\
\\

\begin{itemize}
    \item \textbf{Diseño del sistema} El propósito es establecer un concepto solución el cual describa las principales características operativas físicas y lógicas del producto. % para ello la función global de un sistema se descompone en subfunciones principales.
    \item \textbf{Diseño del dominio específico} - En esta parte se requieren interpretaciones y cálculos más estrictos con el fin de garantizar el correcto desempeño de la función principal.
    \item \textbf{Integración del sistema} -los resultados obtenidos  se integran con el fin de conformar un sistema global. Esta integración permite el análisis de las interacciones entre los componentes, garantizando que el comportamiento conjunto del sistema responda a los objetivos planteados.
    \item \textbf{Validación y Verificación} - El progreso logrado con el diseño debe ser verificado continuamente con base en el concepto de solución  y a los requisitos establecidos.
    \item \textbf{Modelado y análisis del modelo} - Las fases descritas se apoyan a través del uso de modelos y herramientas asistidas por computadora para la simulación del sistema.
\end{itemize}
