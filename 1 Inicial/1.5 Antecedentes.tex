\section{Antecedentes}

\label{Antecedentes}
En esta sección, se presentan diversos trabajos relevantes que sirvieron como referencia para la realización del proyecto. Se enumeran hallazgos obtenidos relacionados con sistemas mecatronicos para armar el cubo rubik, asi como sistemas relacionados con vision artificial sobre lectura de las caras y algoritmos de resolucion.

Los antecedentes se han seleccionado considerando publicaciones previas que operan en condiciones similares con respecto a materiales, equipos y requerimientos del proyecto actual. 


%\newpage
\begin{sidewaystable}
  \scriptsize
  \centering
  \caption{ Síntesis de antecedentes del sistema propuesto (Parte 1)}
  \begin{tabular}{m{5mm} m{25mm} m{45mm} m{40mm} m{15mm} m{18mm} m{12mm} m{14mm}}
    \toprule
    \centering \textbf{No.} & \centering \textbf{Nombre} & \centering \textbf{Descripción} & \centering \textbf{Características} & \centering \textbf{País} & \centering \textbf{Organización} & \centering \textbf{Tipo} & \textbf{Referencia} \\ 
    \midrule
    \centering 1 & \centering Mechanical Automation with Vision: A Design for Rubik’s Cube Solver. & 
    \parbox{45mm}{Sistema autónomo que combina visión por computadora con un mecanismo físico para resolver un cubo Rubik.} & 
    \parbox{40mm}{
        -El sistema utiliza el algoritmo de Kociemba.\\
        -Sistema mecánico automatizado.\\
        -Implementa motores paso a paso (tres en total).\\
        -Emplea YOLOv8 para la detección de colores.\\} & 
    \centering Nepal & \centering Thapathali Campus & \centering Articulo. & \qquad[\cite{chalise2025mechanical}] \\ 
    \midrule
    \centering 2 & \centering Robotic Cuber: A Rubik’s Cube solving robot. & \parbox{45mm}{El proyecto consiste en el diseño y construcción de un robot de cuatro brazos capaz de escanear un cubo Rubik mediante cámara, procesar su estado, y manipular físicamente el cubo para resolverlo.} & 
    \parbox{40mm}{
    -Uso de una cámara web para capturar las caras.\\
    -Basado en el algoritmo de Kociemba.\\
    -Diseño con cuatro brazos tipo garra para sujetar y rotar el cubo.\\
    -Para la configuración rápida, el tiempo promedio fue de ~88.07 s.\\
     }
 & 
    \centering Suecia & \centering KTH Royal Institute of Technology
 & \centering Articulo. & \quad[\cite{gronas2020robotic}] \\
    \midrule
    \centering 3 & \centering How to make CUBOTino micro: The World’s smallest Rubik’s cube solver robot. & \parbox{45mm}{ Manual para construir, configurar y operar el CUBOTino micro, robot más pequeño del mundo capaz de resolver un cubo Rubik de 30 mm.  El robot integra visión artificial, detección de colores y el algoritmo de Kociemba para resolver el cubo en un tiempo promedio a 70 segundos.} & 
    \parbox{40mm}{
    -Diseñado para un cubo Rubik de 30 mm.\\
    -Tamaño muy compacto y bajo costo.\\
    -Usa PiCamera ajustada a corta distancia.\\
    -Dos servos micro metálicos (180--190°).\\
    -Basado en Raspberry Pi Zero2W.\\
    -Implementa el solver de Kociemba.\\
    } & 
    \centering Paises bajos & \centering Instructables Project & \centering Manual. & \qquad[\cite{favero2025cubotino}] \\
    \bottomrule
  \end{tabular}  
\end{sidewaystable}

\begin{sidewaystable}
  \scriptsize
  \centering
  \caption{Síntesis de antecedentes del sistema propuesto (Parte 2)}
  \begin{tabular}{m{5mm} m{25mm} m{45mm} m{40mm} m{15mm} m{18mm} m{12mm} m{14mm}}
    \toprule
    \centering \textbf{No.} & \centering \textbf{Nombre} & \centering \textbf{Descripción} & \centering \textbf{Características} & \centering \textbf{País} & \centering \textbf{Organización} & \centering \textbf{Tipo} & \textbf{Referencia} \\ 
    \midrule
    \centering 4 & \centering Fabricación de prototipo de robot solucionador del cubo de Rubik. & 
    \parbox{45mm}{Trabajo de fin de grado que diseña y fabrica un robot “envolvente” con 6 motores paso a paso para resolver un cubo de Rubik.} & 
    \parbox{40mm}{
        -Arquitectura mecánica (6 DoF): Estructura con seis NEMA-17.\\
        -Kociemba en Raspberry Pi.\\
        -resuelve en menos de 1 minuto.\\
        } & 
    \centering España & \centering ETSI Sevilla & \centering Tesis. & \qquad[\cite{gonzalezaparicio2022rubikrobot}] \\ 
    \midrule
    \centering 5 & \centering Robotic Cuber: A Rubik’s Cube solving robot. & \parbox{45mm}{Trabajo de titulación que diseña un sistema automatizado para detectar y resolver un cubo Rubik 3×3. Integra visión por computadora, control electrónico y una app móvil.} & 
    \parbox{40mm}{
    -Control de servomotores para giros; uso de controlador Pololu de 18 canales.\\
    -App Android/móvil que se comunica por Wi-Fi con el sistema para enviar comandos en tiempo real y activar funciones.\\
    -Detección de colores en HSV, OpenCV y filtrado/descartes para robustez.\\
     }
 & 
    \centering Ecuador & \centering Universidad Católica de Cuenca
 & \centering Tesis. & \quad[\cite{lopezrojas2025rubikapp}] \\
%    \midrule
%    \centering 6 & \centering How to make CUBOTino micro: The World’s smallest Rubik’s cube solver robot. & \parbox{45mm}{ Manual para construir, configurar y operar el CUBOTino micro, robot más pequeño del mundo capaz de resolver un cubo Rubik de 30 mm.  El robot integra visión artificial, detección de colores y el algoritmo de Kociemba para resolver el cubo en un tiempo promedio a 70 segundos.} & 
%    \parbox{40mm}{
%    -Diseñado para un cubo Rubik de 30 mm.\\
%    -Tamaño muy compacto y bajo costo.\\
%    -Usa PiCamera ajustada a corta distancia.\\
%    -Dos servos micro metálicos (180--190°).\\
%    -Basado en Raspberry Pi Zero2W.\\
%    -Implementa el solver de Kociemba.\\
%    } & 
%    \centering Paises bajos & \centering Instructables Project & \centering Manual. & \qquad[\cite{favero2025cubotino}] \\
    \bottomrule
  \end{tabular}  
\end{sidewaystable}

