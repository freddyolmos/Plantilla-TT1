\section{Enfoque mecatrónico}

Para poder desarrollar el \textbf{Sistema mecatrónico de entrenamiento asistido para el aprendizaje del método Fridrich en el cubo Rubik 3×3}, es necesaria la integración sinérgica de cuatro disciplinas: \textbf{Mecánica, Electrónica, Informática y Control (MEIC)}. El objetivo principal de este sistema es preparar automáticamente \textbf{estados específicos del método CFOP (F2L, OLL y PLL)} a partir de un cubo resuelto, aplicando el algoritmo inverso del caso seleccionado y validando el resultado mediante visión artificial. Con ello se permite al usuario practicar de forma repetitiva, rápida y precisa los algoritmos del método Fridrich.

La \textbf{Mecánica} está presente en la estructura del robot, que consiste en un marco central con \textbf{seis mecanismos giratorios} (uno por cada cara del cubo), equipados con \textbf{pinzas o prensas de sujeción} diseñadas en 3D. Estos mecanismos permiten sujetar firmemente el cubo y girar sus caras 90° o 180° con precisión. Adicionalmente, se contempla una \textbf{montura para cámara} que facilita la integración del sistema de visión artificial.

La \textbf{Electrónica} es responsable de la instrumentación y la gestión de energía del sistema. Incluye un \textbf{microcontrolador embebido} (ej. ESP32) para la ejecución básica de movimientos, una \textbf{Raspberry Pi 4/5} como cerebro central que coordina la visión artificial y la interfaz gráfica, drivers de motor, una fuente de poder dedicada y un módulo de cámara. También pueden emplearse \textbf{sensores de posición opcionales} para retroalimentación de los giros y mejorar la confiabilidad.

La \textbf{Informática} se materializa en la aplicación de escritorio que sirve como interfaz hombre-máquina. Desde esta, el usuario selecciona el caso que desea practicar, visualiza el algoritmo asociado y envía el comando al robot. La base de datos del sistema contiene los algoritmos CFOP y sus respectivos inversos, que se transmiten al controlador para ejecutar la preparación del cubo. Además, la Raspberry Pi ejecuta el procesamiento de imágenes mediante \textbf{visión artificial con OpenCV}, validando que el cubo quedó en el estado correcto antes y después de cada secuencia.

Finalmente, el \textbf{Control} regula la secuenciación de los movimientos motores y asegura la correcta ejecución de los giros. A través de una máquina de estados, el sistema interpreta la notación estándar del cubo (R, R', U2, etc.) y la traduce en pulsos precisos para los actuadores. Asimismo, sincroniza los movimientos con el sistema de visión, permitiendo la verificación y corrección en caso de errores. Esto garantiza que cada caso se genere con exactitud y pueda repetirse múltiples veces sin desviaciones.

La ventaja de este enfoque radica en que, al integrar las cuatro disciplinas, se obtiene un \textbf{Sistema Mecatrónico docente} superior a los modelos precedentes. Mientras otros robots solo resuelven el cubo por completo, este sistema se enfoca en la \textbf{automatización de la preparación de casos específicos}, lo que acelera el aprendizaje, reduce la frustración y facilita la práctica intensiva. Además, su carácter de código abierto y modular permite futuras expansiones, como añadir nuevos modos de entrenamiento o adaptar el sistema a otros métodos de resolución.