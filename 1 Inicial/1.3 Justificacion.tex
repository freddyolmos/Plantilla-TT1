\section{Justificación}

La resolución de los rompecabezas ocupa un lugar privilegiado en la literatura pedagógica por su capacidad de estimular procesos cognitivos complejos. Algunos estudios sobre \emph{juegos matemáticos} muestran que desafíos como el cubo Rubik generan entornos donde la abstracción se materializa y la motivación intrínseca se incrementa, favoreciendo la comprensión de patrones y el desarrollo del pensamiento lógico \cite{SilvaMera2024}. Entre los beneficios más citados se encuentran la mejora del razonamiento espacial, la memoria de trabajo y la capacidad para planificar estrategias. Estas competencias, esenciales en ámbitos STEM, se potencian aún más cuando la actividad se enmarca en un contexto lúdico que refuerza la confianza y la perseverancia del estudiante.

Ahora bien, pasar del método básico de resolución del cubo al método Fridrich (CFOP) supone memorizar más de setenta algoritmos y reproducir situaciones muy específicas para su práctica deliberada. El estudiante se ve obligado a “preparar” el cubo manualmente antes de cada intento, un proceso tedioso que diluye el tiempo de estudio efectivo y genera frustración. Esta brecha entre el valor pedagógico del cubo y la ineficiencia del entrenamiento avanzado justifica la búsqueda de una herramienta que libere al alumno de la preparación mecánica, permitiéndole concentrarse en la adquisición y perfeccionamiento de los algoritmos.

La mecatrónica —concebida como integración sinérgica de mecánica, electrónica, informática y control— ofrece la plataforma ideal para cerrar esa brecha. Su enfoque \emph{transdisciplinario} pone en diálogo permanente a las distintas ingenierías, generando soluciones donde la frontera entre disciplinas se diluye para dar paso a la sinergia \cite{AquinoRobles2019}.  Concebir un robot entrenador del cubo Rubik implica diseñar un mecanismo de accionamiento preciso (mecánica), seleccionar motores y electrónica de potencia (electrónica), programar algoritmos inversos y una interfaz intuitiva (software) y, finalmente, sincronizar todos los subsistemas mediante estrategias de control robustas (automatización). Cada etapa del proyecto —modelado cinemático, selección de actuadores, control de movimiento, visualización de algoritmos— exige la participación coordinada de estas áreas, materializando el carácter multidisciplinario que define la formación del ingeniero mecatrónico.

En términos formativos, el proyecto actúa como catalizador de las competencias profesionales que demanda la Industria 4.0: capacidad de integración tecnológica, pensamiento sistémico y orientación a la innovación \cite{AquinoRobles2019}.  Al enfrentarse al ciclo completo de diseño-construcción–validación, el estudiante aplica conocimiento teórico en un problema real, fortalece habilidades de gestión de proyectos y consolida un portafolio con alto valor académico y laboral.

En síntesis, justificar el desarrollo de un robot entrenador de casos CFOP descansa en dos pilares complementarios:  
\begin{enumerate}
    \item \textbf{Cognitivo–educativo}. Ampliar los beneficios demostrados del cubo Rubik, eliminando la barrera mecánica que frena la práctica avanzada y optimizando el proceso de aprendizaje significativo.  
    \item \textbf{Tecnológico–profesional}. Mostrar la mecatrónica como respuesta natural y multidisciplinaria a un problema concreto, a la vez que se potencia la formación integral del estudiante y se aporta una herramienta transferible a contextos educativos y competitivos.  
  \end{enumerate}

De esta manera, el proyecto no solo atiende una necesidad pedagógica evidente, sino que también se alinea con las tendencias de innovación y con el perfil de competencias que la ingeniería mecatrónica promueve en el siglo XXI.