\section{Organización del documento}

El presente documento se organiza en cinco bloques principales:

\textbf{Capítulo 1. Introducción}. Presenta el enfoque mecatrónico del proyecto, la
definición del problema, la justificación, los objetivos y los antecedentes relevantes para
el desarrollo del robot entrenador CFOP.

\textbf{Capítulo 2. Desarrollo del proyecto}. Integra el marco referencial, el diseño del
sistema, la implementación y el análisis de resultados. En este bloque se documenta la
trazabilidad entre requerimientos, decisiones de diseño y evidencia técnica.

\textbf{Capítulo 3. Cierre}. Incluye conclusiones y recomendaciones de trabajo a futuro,
vinculadas al cumplimiento de objetivos y a las mejoras del sistema.

\textbf{Capítulo 4. Material adicional}. Reúne referencias bibliográficas, apéndices,
anexos y glosario para respaldar la reproducibilidad del proyecto.

Finalmente, la estructura completa sigue la secuencia institucional de los reportes de TT,
priorizando claridad técnica y validación por evidencia.
